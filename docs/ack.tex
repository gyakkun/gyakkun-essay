%%
% 致谢
% 谢辞应以简短的文字对课题研究与论文撰写过程中曾直接给予帮助的人员(例如指导教师、答疑教师及其他人员)表示对自己的谢意,这不仅是一种礼貌,也是对他人劳动的尊重,是治学者应当遵循的学术规范。内容限一页。
% modifier: 黄俊杰
% update date: 2017-04-15
%%

\chapter{致谢}

感谢指导老师cjh副教授, 您的悉心指导成就了这份毕业设计。

此外首先要感谢的是17级软件工程专业的howardlau同学, 它是本文所描述的资源分享站(21weeks.icu)的站长, 是整个项目的实际发起者, 建站过程中许多困难的任务都由他一手承担。可以说没有他就没有这份毕业设计, 我可能就毕不了业了(笑)。其次要感谢SYSU IPv6交流群的首任群主, 16级计算机科学信息安全专业的jipeng同学。自从大二认识以来, 他就是我朝着推广IPv6协议应用这一目标的路上最坚定的伙伴。他对本文提出了诸多重要的意见和建议, 是笔者本文写作时的首席顾问。感谢SYSPT站务群的全体管理员, robinWongM同学对数据库迁移具有不可替代的贡献, MilkFather同学对本地化工作同样有十分重要的贡献, 同时还负责前端部分细节的调整。haswelliris师兄作为毕业生形象大使参与了站务建设, 为本站的存续提供了重要的指导性意见。感谢SYSU IPv6交流群全体群友, 特别感谢LGA1150师兄以及haswelliris师兄, 你们为群里带来的大量技术干货以及对新人群友的热心帮助, 让数度涅槃的SYSU IPv6交流群至今仍具有蓬勃的生机。

\textbf{Sincerely thanks to HDVinnie, your great work makes this essay possible.}

此外还要感谢wyf、oyry、lt、lzh、jbh等ASC19校队的全体队友, 以及haswelliris、fgn、chonor、Alcanderian四位负责训练的师兄。SYSU ASC19校队为学校争得的荣誉是我本科生涯最重要的回忆之一。感谢大学以来遇到的所有老师, 你们让我从对计算机一知半解到如今能够顺利从事相关工作。感谢所有大学同学, 你们让我的大学生活变得多姿多彩。感谢我的舍友, 本科四年间朝夕相处, 希望在将来也能与你们一路相伴, 守望前行。感谢母校的一切。在这里驻足的五年, 有泪有笑, 有苦有乐, 这里必将是人生长路上无可替代的驿站。我将永远保持对母校的感恩。

感谢艇菊、夏娜、大/中/小王子等[U2CSM][BD外挂字幕群]群友, 你们的嬉笑日常为我带来了无数欢乐。感谢离校期间提供食宿的表哥以及他家的三只小猫。

感谢成长路上陪伴过我的友人。lhr: 你还好吗, 多年未联系, 不知安否。lookatme: 希望你在佛山的工作能诸事顺利。谢楚: 求学不易, 直博路上必有艰难险阻, 愿你能从容应对; 从医不易, ``行医者无疆,虽千万里,吾往矣'', 而今病毒肆虐, 寰球震动, 愿你学有所成, 将来定能救死扶伤, 妙手回春。sya: 希望你能走出阴影, 重回正轨。zl: 那通电话真的吓到我了, 希望你我都能找到属于自己的幸福!

最后感谢我的父母, 你们数十年如一日的辛劳把我培养成人, 养育之恩, 无以为报。
